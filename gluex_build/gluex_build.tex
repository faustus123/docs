\documentclass{article}

\usepackage{graphicx}

\begin{document}

\section{Introduction}
There are three fundamental areas of concern that make up all software
systems. They are:

\begin{enumerate}
\item a directory structure
\item a build system
\item a version management system
\end{enumerate}

They are all related, aspects of one affects aspects of each of the others.

Having a standard system makes collaborative work more efficient,
especially when one needs help from others to solve a problem.

\section{The Packages}

There are several software components that are needed to build and use
GlueX software. Most of them are assumed to be provided by the native
operating system or distribution, but there are some that have to be
built by the GlueX user. They are:

\begin{enumerate}
\item build\_scripts: scripts to manage building and the shell environment
\item Xerces-C: for reading XML files
\item CERNLIB: to support GEANT 3 simulations
\item ROOT: general purpose HENP toolkit
\item EVIO: CODA format data handling library
\item CCDB: Calibration Constants Database
\item JANA: event-based analysis framework
\item HDDS: detector geometry specification library 
\item sim-recon: simulation and reconstruction for GlueX
\end{enumerate}

Detailed description of these packages will not be given here; please
see the GlueX Offline Software wiki page for more information.

There are in general multiple versions (releases) of each of these
packages and it is often convenient to have access to more that one
version of a package built and available for use. In addition some
packages depend on one or several others for libraries and include
files.

\section{The Directory Structure}\label{section:directory}

The VMS directory structure supports multiple versions of each
package. For an example see Fig.~\ref{fig:directory-tree}. In the
figure, ``gluex\_top'' is a generic name, each installation may choose
a different directory name. VMS looks for the name of this directory
in the environment variable {\tt GLUEX\_TOP}.

\begin{figure}
\begin{verbatim}
                    gluex_top
                    |-- build_scripts
                    |-- cernlib
                    |   `-- 2005
                    |-- evio
                    |   |-- evio-4.2
                    |   `-- evio-4.3.1
                    |-- hdds
                    |   |-- hdds-3.1
                    |   `-- hdds-3.2
                    |-- jana
                    |   |-- jana_0.7.2
                    |   `-- jana_0.7.3
                    |-- root
                    |   |-- root_5.34.04
                    |   `-- root_5.34.26
                    |-- sim-recon
                    |   |-- sim-recon-1.2.0
                    |   `-- sim-recon-1.3.0
                    `-- xerces-c
                        |-- xerces-c-3.1.1
                        `-- xerces-c-3.1.2
\end{verbatim}
\caption{The directory structure.}\label{fig:directory-tree}
\end{figure}

Under gluex\_top, each package has its own directory and for each
package directory one or more specific versions of that package are
built.

\section{Scripts to Support the VMS}

All scripts and makefiles to support VMS are found the in the {\tt
  build\_scripts} directory. At JLab, the full path is
$${\tt /group/halld/Software/scripts/build\_scripts}.$$
The directory can also checked out from the Subversion repository:
$${\tt https://halldsvn.jlab.org/repos/trunk/scripts/build\_scripts}.$$

\section{The Build System}

Each of the packages have their own native build system and each build
system has its own set of details that have to be understood. In
addition, the technology used to do the build varies from system to
system. It may be make, imake, cmake, Scons, or something else. The
VMS system makes a choice of build options for each package so that
the user need not master these details.

\subsection{The Makefiles}

The VMS build system is implemented in GNU Make. These makefiles
invoke the package-specific build system. There is a ``package
makefile'' for each package ({\it e.~g.}, Makefile\_jana,
Makefile\_sim-recon). Invoking make with a package makefile will build
that package with the home directory placed in the current working
directory.

Complete builds are orchestrated by Makefile\_all. The highest-level
targets of Makefile\_all, shown with their depedencies are:

\begin{verbatim}
all: env xerces_build cernlib_build root_build clhep_build \
     geant4_build gsl_build evio_build ccdb_build jana_build \
     hdds_build sim-recon_build

gluex: env xerces_build cernlib_build root_build clhep_build \
       evio_build ccdb_build jana_build hdds_build sim-recon_build

gluex_jlab: env xerces_build root_build clhep_build evio_build \
            ccdb_build jana_build hdds_build sim-recon_build
\end{verbatim}

The {\tt all} target builds all packages that Makefile\_all knows
about. The {\tt gluex} target builds only the packages necessary for
GlueX work. The {\tt gluex\_jlab} target is the same as {\tt gluex}
except that it does not include cernlib\_build (useful for JLab public
builds).

Each of the individual package targets ({\it e.~g.}, evio\_build and
hdds\_build) use the corresponding package makefile. Directories are
created and the package makes are executed in way that gives the
directory structure described in Section~\ref{section:directory}. Of
course, each individual package build target can be invoked directly.

\subsection{The Package Makefiles}

Each of the package makefiles is sensitive to environment variables that control which version of the package to build. The makefiles themselves take care of obtaining the source code.

Each packages respects a version-specifying environment variable. Here is an example of how they might be set in the C-shell:

\begin{verbatim}
setenv JANA_VERSION 0.7.2
setenv SIM_RECON_VERSION 1.2.0
setenv HDDS_VERSION 3.2
setenv CERNLIB_VERSION 2005
setenv XERCES_C_VERSION 3.1.1
setenv CLHEP_VERSION 2.0.4.5
setenv ROOT_VERSION 5.34.26
setenv CCDB_VERSION 1.05
setenv EVIO_VERSION 4.3.1
\end{verbatim}

In each case there is standard system for distributing tarballs marked
with the version name. Each package has different conventions, but the
package makefiles have that knowledge of the appropriate convention
coded in. Also, the name of home directory created depends on the name
that appears in the tarball (with exceptions as mentioned in
Section~\ref{section:directory-tags}. Note that the version variable
can be set on the make command line as well.

Some packages can be checked out from a Subversion repository. In
these cases, the version variable is ignored. JANA, HDDS, and
sim-recon have this support. So for example:

\begin{verbatim}
setenv HDDS_URL https://halldsvn.jlab.org/repos/trunk/hdds
\end{verbatim}

will cause the HDDS makefile to check out the trunk version. The names
of the variables for JANA and sim-recon are {\tt JANA\_URL} and {\tt
  SIM\_RECON\_URL} respectively.

\section{Setting Up the Shell Environment}

Facility is provided for setting environment variables necessary both
for building the software and for using it. Both Bourne-shell-like and C-shell-like shells are supported, but real testing has only been done with bash and tcsh. In the following all examples will be appropriate for bash. Note that whenever a script like foo.sh is mentioned, there is also a foo.csh in the build-scripts directory. 

\section{The gluex\_install System}

Independent of the build\_scripts directory, the gluex\_install system
uses build\_scripts to create a complete install of GlueX software
from scratch. This is especially useful for new machines.

No interaction from the user should be required to get a successful
build. The only assumption made is that the basic packages that come
in a minimal install are present. The definition of minimal
depends on the installation. In all cases, the distribution was tested
by first installing from a DVD or CD iso image. Typically, the ``live
DVD'' version was chosen since that installs the smallest number of
packages.

The scripts have been tested on the following distributions listed in
Table~\ref{table:tested-distributions}.

\begin{table}
$$
\begin{tabular}{|l|l|}
\hline
\bf Distribution & \bf Package Type \\
\hline
CentOS & RedHat \\
Scientific Linux & RedHat \\
Ubuntu & Debian \\
Fedora & RedHat \\
LinuxMint & Debian \\
openSUSE & RedHat \\
RedHat Enterprise & RedHat \\
\hline
\end{tabular}
$$
\caption{Gluex\_install tested distributions.}\label{table:tested-distributions}
\end{table}

\subsection{Installation Steps}

Root access is required for steps (1) and (2).

\begin{enumerate}

\item {\bf System Update}. It is recommended that you update your
  system to the latest versions of of all system supplied
  software. For RedHat-like distributions you do a ``yum update''. For
  Debian-like systems you do a ``apt-get update''.

\item {\bf Get the Scripts}. A tar file with these scripts is
  available at https://halldweb.jlab.org/dist/gluex\_install.tar . You
  can also do a subversion check out of the latest version at
  https://halldsvn.jlab.org/repos/trunk/scripts/gluex\_install . Note
  that most distributions do not have subversion in their minimal set
  of packages.

\item {\bf Prerequisites: gluex\_prereq\_<distribution>.sh}. The
  prerequisites script installs packages from the distribution
  repository necessary for the GlueX build. As such, it must be
  executed by root. In addition it makes some symbolic links in system
  directories that are necessary for the cernlib build. These scripts
  are specific to particular distributions. You must run this
  script from inside the ``gluex\_install'' directory created when you
  get the scripts (see Get the Scripts, step 2).

\item {\bf Subversion test: svn\_touch.sh}. Simply does an ``l'' of the
  Hall D and 12 GeV subversion repositories at JLab, both as a test
  and to dispense with interactive prompts asking about certificates
  from the servers. Respond with ``p'', to permanently accept the
  certificates. These prompts would otherwise hang the build. If you
  are not prompted, you already recognize the certificates. The script
  is distribution independent.

\item {\bf Install: gluex\_install.sh}. Creates a directory,
  ``gluex\_top'', in the current working directory to house the build,
  sets up an environment, downloads all source files, and builds all
  libraries and executables needed to run GlueX software. The install
  assumes a directory structure that accomodates multiple versions of
  the GlueX packages if they are needed later. The script is
  distribution independent.

\end{enumerate}

\subsection{Using the Build}

After the build is complete, there are two files in the gluex
directory, setup.sh and setup.csh, that can be used to set-up the
complete GlueX environment under Bourne-like shells or C-like shells
respectively.

\end{document}


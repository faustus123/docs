\documentclass[12pt]{article}

\title{Progress in Hall D Offline Software, 2015}
\author{Mark Ito, David Lawrence, Curtis Meyer}
\date{February 9, 2016}

% global parameters
\textwidth=6.5in
\oddsidemargin=0in % use built-in offset of 1 inch for left margin
\evensidemargin=0in % ditto for even pages
\textheight=9in
\topmargin=0in
\headheight=0in % no headers in this document
\headsep=0in

\begin{document}

\maketitle

We summarize new efforts in GlueX offline software development in the year just past.

\section{Conversion from GEANT 3 to Geant4}

Work continues on converting our GEANT 3 implementation of detector
simulation to Geant4. The geometry definition complete in the new
system. Here an XML-based specification is read in at run time and
appropriate classes instantiated. Work continues on implementing
detector hit generation. One of the main benefits to the new system
will be the ability to run multi-threaded. Scaling of execution speed
with threads has been demonstrated.

There are two implementations at present: one for the standard GlueX
program (HDGeant4)\cite{g4-gluex} and one for the Charged Pion
Polarizability Experiment (CPPsim)\cite{g4-cpp}.

\section{Adoption of SWIF}

We have incorporated use of Scientific Computing's system for managing
collections of related farm jobs, SWIF, into all of our large-scale
computing projects. These include the bi-weekly trains doing reconstruction of
recent data, simulation studies, and data challenges.

\section{Electromagnetic Background Simulation}

We have developed a method for incorporating a separately generated
library of out-of-time electromagnetic background events into a sample
of simulated physics events. This leverages the existing background
generator, which takes into account a coherent photon beam and beam
collimation to produce out of time hits. This method saves time by
avoiding generation of E\&M background on an event-by-event basis.

\section{EventStore}

Work continues on implementing EventStore (from CLEO) into GlueX.
The system uses sets of independent event indices to
allow access to a common reconstructed data set based on criteria that
vary from index to index. This function is usually done by skimming
data sets to produce specialized streams in independent data sets, but
that method often results in many events contributing to multiple
streams with a net increase in the size of the reconstructed data
set. Compact reconstructed data, beyond being efficiently stored, is
easier to distribute to off-site institutions.

\section{Conversion to Git}

This past summer we transitioned from using Subversion for source code
version control to using Git. We had a program of explaining the new
system to the collaboration, designing the style of usage we would
adopt, and providing documentation to help the new users. There has
been marked improvement in communication about changes to the code
using the tools available via GitHub\cite{gluex-github}.
This completes a response to a
recommendation from one of the Software Reviews.

We have also implemented a system where pull requests on GitHub generate a build
at JLab using the branch proposed in the pull request. We can now
judge whether a change will compile and link before it is merged into
the master branch.

\section{Spring 2015 Simulations Complete}

We completed a set of simulations to support analysis of data taken in
the Spring of 2015. Thirty thousand jobs were run successfully on the
farm for this effort.

\section{Software Distribution and Building}

A system for managing our various software packages,
both those externally provided
and those developed in-house, was enhanced and documented during the
year. The goal is to insulate the user from the need to the master
details of building each of several software packages as well as from
the details of setting up a working environment. Multiple versions of
each of several packages can be maintained simultaneously. Particular
combinations of package versions can be specified succinctly in an XML
configuration file and this file can be used both to guide a complete
build of all needed packages and to set up the shell environment to
use the resulting build.

In a parallel development, one of our members has been working on a
system focused on supporting multiple builds of particular packages to
compare results among them. The system is used to create binary
distributions of all packages needed to run and develop GlueX
software.

\section{Calibration Challenge}

Work is in progress on consolidating and automatizing all known
calibration activities. Work will proceed in several passes, each one
dependent on the results of the previous pass. The goal is production
of calibration constants soon after experimental data is taken. A
framework has been developed and several collaborators have started
contributing the working pieces in the form of run-time-loaded plugins.

The calibration effort in general is going well with all detector
systems nearly at design resolution, if not better.

\section{Clang-based C++ Code Analyzer}

We are now doing nightly generation of reports on questionable passages
of our C++ code base using the Clang/LLVM compiler suite. A web page
is generated; problem areas are displayed and explanations
for why these lines of code were flagged are given.

\section{Upcoming Data Taking Run}

We are planning to have the software components in place to be able to
calibrate and process all data during the run itself.

\begin{thebibliography}{9}

\bibitem{g4-gluex} https://github.com/rjones30/HDGeant4

\bibitem{g4-cpp}
  https://halldsvn.jlab.org/repos/trunk/Experiments/PionPolarizability/src/CPPsim

\bibitem{gluex-github}
  https://github.com/orgs/JeffersonLab/teams/gluex/repositories

\end{thebibliography}

\end{document}

\documentclass[12pt]{article}

\title{Progress in Hall D Offline Software, 2015}
\author{Mark M.\ Ito}
\date{February 4, 2016}

% global parameters
\textwidth=6.5in
\oddsidemargin=0in % use built-in offset of 1 inch for left margin
\evensidemargin=0in % ditto for even pages
\textheight=9in
\topmargin=0in
\headheight=0in % no headers in this document
\headsep=0in

\begin{document}

\maketitle

\section{Adoption of SWIF}

We have incorporated use of Scientific Computing's system for managing
collections of related farm jobs, SWIF, into all of our large-scale
computing projects. These include the bi-weekly reconstruction of
recent data, simulation studies, and data challenges.

\section{Electromagnetic Background Simulation}

We have developed a method for incorporating a separately generated
library of put-of-time electromagnetic background events into a sample
of simulated physics events. This leverages the existing background
generator, which takes into account a coherent photon beam and beam
collimation to produce out of time hits. This method saves time by
avoiding generation of E\&M background on an event-by-event basis.

\section{EventStore}

Work continues on a system for adopting the CLEO EventStore system for
use by GlueX. The system uses sets of independent event indices to
allow access to a common reconstructed data set based on criteria that
vary from index to index. This function is usually done by skimming
data sets to produce specialized streams in independent data sets, but
that method often results in many events contributing to multiple
streams with a net increase in the size of the reconstructed data
set. Compact reconstructed data, beyond being efficiently stored, is
easier to distribute to off-site institutions.

\section{Conversion to Git}

This past summer we transitioned from using Subversion for source code
version control to using Git. We had a program of explaining the new
system to the collaboration, designing the style of usage we would
adopt, and providing documentation to help the new users. There has
been marked improvement in communication about changes to the code
using the tools available via GitHub. This completes a response to a
recommendation from one of the Software Reviews.

We have also implemented a system where pull-request generate a build
at JLab using the branch proposed in the pull-request. We can now
judge whether a change will compile and link before it is merged into
the master branch.

\section{Spring 2015 Simulations Complete}

We completed a set of simulations to support analysis of data taken in
the Spring of 2015. Thirty thousand jobs were run successfully on the
farm for this effort.

\section{Software Distribution and Building}

A system our various software packages, both those externally provided
and those developed in-house, was enhanced and documented during the
year. The goal is to insulate the user from the need to the master
details of building each of several software packages as well as from
the details of setting up a working environment. Multiple versions of
each of several packages can be maintained simultaneously. Particular
combinations of package versions can be specified succinctly in an XML
configuration file and this file can be used both to guide a complete
build of all needed packages and to set up the shell environment to
use the resulting build.

In a parallel development, one of our members has been working on a
system focused on supporting multiple builds of particular packages to
compare results among them. The system is used to create binary
distributions of all packages needed to run and develop GlueX
software.

\section{Calibration Challenge}

Work is in progress on consolidating and automatizing all known calibration activities. Work will proceed in several passes, each one dependent on the results of the previous pass. The goal is production of calibration constants soon after experimental data is taken. A framework has been developed and several collaborators have started contributing the working pieces.

\section{Clang-based C++ Code Analyzer}

We are now doing night generation of reports on questionable passages of our C++ code base using the Clang/LLVM compiler suite. A web page is generated; problem area are displayed and explanations for why code was flagged are given.

\end{document}
